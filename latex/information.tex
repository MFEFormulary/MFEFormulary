\chapter*{Information}
\addcontentsline{toc}{chapter}{Information} The following sections
contain important information regarding the development, use, and
distribution of the MFE Formulary. The most up-to-date contact
information for the authors of the MFE Formulary is:
\begin{itemize}
  \item \textbf{Email:} \href{mailto:mfe_formulary@mit.edu}{mfe\_formulary@mit.edu}
  \item \textbf{Web:} \href{http://www.psfc.mit.edu/research/MFEFormulary}{http://www.psfc.mit.edu/research/MFEFormulary}
  \item \textbf{GitHub:} \href{https://github.com/MFEFormulary/MFEFormulary}{https://github.com/MFEFormulary/MFEFormulary}
\end{itemize}

\section*{Licensing}
The Formulary has been released under open source licenses in order to
maximize its distribution and utility to the plasma physics and
magnetic fusion communities. The hardcopy printed book and digital PDF
format have been released under the
\href{http://creativecommons.org/licenses/by-sa/4.0/legalcode}{Creative
  Commons SA 4.0 license}. The LaTex source code and build system has
been released under the
\href{http://www.gnu.org/licenses/gpl.html}{GNU GPL v3.0} and is
freely available from our GitHub repository on the web. Both licenses
permit copying, redistributing, modifying, and deriving new works
under the aforementioned license. You are encouraged to distribute
this document as widely as possible under the terms provided in the
license above. You are also encouraged to fork or clone our GitHub
repository, make your own contributions to the LaTeX codebase, and
submit a pull request for inclusion of your contribution to the
Formulary. (Email to works as well.) Finally, you are welcome to use
any part of the hardcopy or electronic copy of the Formulary or its
LaTeX code in your own projects under the terms of the licenses above.

\section*{Contributors}
The following is a list of people who have contributed their time,
effort, and expertise. Their contributions have been critical to
continually improving the accuracy and content of the Formulary. Please
consider contributing your expertise either through our GitHub
repository or email!

\begin{multicols}{2}
\begin{itemize}
\item Michael Bongard
\item Dan Brunner
\item Mark Chilenski
\item Samual Cohen
\item Luis Delgado-Aparicio
\item Chi Gao
\item Greg Hammett 
\item Alex Tronchin-James
\item Anne White
\item Dennis Whyte 
\end{itemize}
\end{multicols}


\section*{Disclaimer}
The MFE formulary is by no means complete or error-free. In fact, the
statistical laws of the universe essentially guarantee that it has
grievous errors and glaring omissions. Thus, we welcome suggestions
for additional material, improvements in layout or usability, and
corrections to the pesky errors and typos we have tried so hard to
eliminate. Even better, we encourage you to obtain the source code
from our GitHub repository (see above), make the corrections, and then
submit them to us via pull request. You can also reach us by email
(see above).\\

\noindent
Despite vigorous proofreading, no guarantee is provided by the authors
as to the acuracy of the material in the MFE Formulary. The authors
shall have no liability for direct, indirect, or other undesirable
consequences of any character that may result from the use of the
material in this work. This may include, but is not limited to, your
analysis code not working or your tokamak not igniting. By your use of
the Formulary and its contents, you implicitly agree to absolve the
authors of liability. The reader is encouraged to consult the original
references in all cases and is advised that the use of the materials
in this work is at his or her own risk.

\section*{Acknowledgments}
The authors would like to thank Dan Brunner, Chi Gao, Christian
Haakonsen, Dr. John Rice, Prof. Anne White, Prof. Dennis Whyte (all of
MIT), and Dennis Boyle and Dr. Samuel Cohen (PPPL) for their
encouragement, feedback, and proofreading. We would also like thank
Heather Barry, Prof. Richard Lester, and Prof. Miklos Porkolab of MIT
for their support.\\

\noindent
During the writing of the Formulary, the authors were at various times
supported by the following funding agencies, to which we would like to
extend our gratitude: the MIT Department of Nuclear Science and
Engineering, the MIT Plasma Science and Fusion Center, U.S DoE Grant
DE-FG02-94ER54235, U.S. DoE Cooperative Agreement DE-FC02-99ER54512,
and U.S. ORISE Fusion Energy Sciences Program.
