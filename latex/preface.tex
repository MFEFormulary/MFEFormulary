\chapter*{Preface}
\addcontentsline{toc}{chapter}{Preface}
\noindent
The guiding principle behind the MFE formulary (the ``Formulary''
hereafter) is to provide a comprehensive reference for students and
researchers working in the field of magnetic confinement fusion. It is
a product of the authors' frustration with searching dozens of
textbooks while studying for the MIT doctoral qualifying exams.\\

\noindent
The Formulary consists of three broad sections. Chapters 1--2 cover
the mathematics, fundamental units, and physical constants relevent to
magnetic fusion. Chapters 3--9 cover the basic physics of
thermonuclear fusion plasmas, beginning with electrodynamics as a
foundation and developing single particle physics, plasma parameters,
plasma models, plasma transport, plasma waves, and nuclear
physics. Chapters 10--13 cover the physics of toroidally confined
plasmas, the fundamentals of magnetic fusion energy, and the
parameters for the major magnetic fusion devices of the world.\\

\noindent
Much of the content of the Formulary has been derived from an original
source, such as peer-reviewed literature, evaluated nuclear data
tables, or the pantheon of ``standard'' mathematics and physics
textbooks commonly used in magnetic fusion energy. References are
given immediately following the cited item in superscript form as
``a:b'', where ``a'' is the citation number of the reference and ``b''
is the page number. Full bibliographic entries for all references may
be found at the end of the formulary. In addition to providing
transparency, this unique feature transforms the Formulary into a
gateway to a deeper understanding of the critical equations,
derivations, and physics for magnetic fusion energy. \\


\noindent
Ultimately, we hope that this work is useful to all those trying to
make magnetic fusion energy a reality.\\
\begin{center}
Z \& Y
\end{center}
\vfill
\pagebreak

