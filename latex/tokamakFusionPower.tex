\chapter{Tokamak Fusion Power}
In this chapter, all units are SI with the exception of temperature
and energy, which are defined in the historical units of eV
(electron-volts).\\

\noindent
$n$ is the plasma density; $n_{20}=n/10^{20}$; $n_0 = n_e \approx n_i$\\
$T$ is the plasma temperature; $T_\mathrm{keV}$ = $T$ in units of kiloelectron-volts\\
$p$ is the plasma pressure\\
$\nu$ is the radialprofile peaking factor\\
$P$ is a power density\\
$S$ is a total power\\
$U$ is a total energy\\
$E$ is the nuclear reaction energy gain\\
$e$ is the elementary electric charge\\
$q$ is the total particle charge\\
$Z$ is the particle atomic (proton) number\\
$e$ and $i$ subscripts refer to electrons and ions, respectively\\
$D$ and $T$ refer to deuterium and tritium, respectively\\
$\kappa$ is the plasma elongation; $\kappa$=b/a\\
$a$,$b$, and $R_0$ are the 2 minor and major radii of a toroidal plasma\\
$V$ is the volume of the plasma\\

\section{Definitions}
\index{Power densities}
\noindent
Fusion power density \scite{wesson}{8}
\fla{P_\mathrm{fusion} &= n_D n_T \langle\sigma v\rangle_{DT} E_\mathrm{fusion} = \frac{1}{4}n_e^2\langle\sigma v\rangle_{DT} E_\mathrm{fusion} &}

\noindent
Fusion power \scite{wesson}{22}
\fla{S_\mathrm{total} &= \frac{\pi}{2}E_\mathrm{fusion} \!\!\int \limits_{\substack{plasma\\ cross \\ section}}\!\! n^2\langle\sigma v\rangle_{DT} R\,dS & \\
                      &= \frac{0.15}{2\nu +1}Rab\left(\frac{n}{10^{20}}\right)^2T_\mathrm{keV}^2 \hspace{0.5cm} \hspace{0.5cm} \mathrm{[MW]} &}
\indent
where it has been assumed that:
\begin{eqnarray*}
  &\text{Pressure profile: }& nT = \hat{n}\hat{T}\left(1-\frac{r^2}{\tilde{a}^2}\right)^\nu \\
  &\text{Reaction rate: }& \langle\sigma v\rangle_\mathrm{DT} \approx 1.1\times10^{-24}T_\mathrm{keV}^2 \\
  &\text{Plasma cross section: }& \tilde{a} = (ab)^{1/2}
\end{eqnarray*}

\noindent
Alpha power density \scite{wesson}{10}
\fla{P_\mathrm{\alpha} &= n_D n_T \langle\sigma v\rangle_{DT} E_\alpha = \frac{1}{4}n_e^2\langle\sigma v\rangle_{DT} E_\alpha  \approx (1/5)P_\mathrm{fusion}&}

\noindent
Neutron power density \scite{wesson}{10}
\fla{P_\mathrm{neutron} &= n_D n_T \langle\sigma v\rangle_{DT} E_\mathrm{neutron} = \frac{1}{4}n_e^2\langle\sigma v\rangle_{DT} E_\mathrm{neutron} \approx (4/5)P_\mathrm{fusion}&}

\noindent
Ohmic heating power density \scite{wesson}{240}
\fla{P_\mathrm{ohmic} &\approx \eta J_\mathrm{plasma}^2 &}

\index{Stored plasma energy}
\noindent
Stored energy in confined plasma \scite{wesson}{9}
\fla{ W &= \int 3nT\,d\tau = 3\langle nT\rangle V &}

\index{Energy confinement time}
\noindent
Definition of energy confinement time \scite{wesson}{9}
\fla{ \tau_E &= \frac{\text{Stored energy in the confined plasma}}{\text{Power lost from the confined plasma}} = \frac{W}{S_\mathrm{loss}} &}

\index{Conduction power losses}
\noindent
Power loss from a confined plasma due to conduction \scite{wesson}{9-10}
\fla{P_\mathrm{conduction} &= \frac{3nT}{\tau_E} &}

\index{Bremsstrahlung power losses}
\noindent
Power loss from a confined plasma due to bremsstrahlung radiation \scite{wesson}{227-228}
\fla{P_\mathrm{bremsstrahlung} &\approx (5.35\times10^{-37})Z^2 n_e n_i T_\mathrm{keV}^{1/2} \hspace{0.5cm} \text{[W m$^{-3}$]}  &}


\index{Power balance}
\section{Power Balance in a D-T Fusion Reactor}
Confined fusion plasmas are not in thermal equilibrium, and,
therefore, power must be balanced in a steady-state tokamak reactor.
Power that is lost from the confined plasma due to conduction,
radiation and other mechanisms must be continuously replenished by
alpha particle and auxilliary heating mechanisms.

\noindent
\fla{ 0 &= \left(P_{\mathrm{alpha}} + P_{\mathrm{auxilliary}}\right) - \left(P_{\mathrm{conduction}} + P_{\mathrm{bremsstrahlung}} + \ldots\right) &}
\indent 
where 
\fla{\indent &P_{\mathrm{auxilliary}} = P_{\mathrm{ohmic}} + P_{\mathrm{ICH}} + P_{\mathrm{ECH}} + P_{\mathrm{neutral~beam}} + \ldots &}

\subsection{Impurity Effects on Power Balance}
The fractional impurity densities $f_j=n_j/n_0$ in the plasma core cause:
\begin{enumerate}
  \item{Modified quasi-neutrality balance \scite{wesson}{36}
    \fla{ n_e &= n_D + n_T + \sum\limits_jZn_j &}}
  \item{Increased radiated power loss~~\cite{authors}
    \fla{ P_{\mathrm{bremsstrahlung}} &\approx (5.35\times10^{-37})n_e^2T_\mathrm{keV}^{1/2}Z_\mathrm{eff} \hspace{0.5cm} \text{[W m$^{-3}$]} &}}
  \item{Dilution of fusion fuel~~\cite{authors}
    \fla{ P_{\mathrm{alpha}} &= \frac{1}{4}n_e^2(1-\sum\limits_jf_jZ_j)^2\langle\sigma v\rangle E_\alpha &}}
\end{enumerate}

\index{Q! Fusion}
\subsection{Metrics of Power Balance}

\index{Physics gain factor}
\noindent
The physics gain factor for D-T plasma \scite{wesson}{12}
\fla{ Q_\mathrm{phys} &= \frac{\frac{1}{4}n_e^2\langle\sigma v \rangle E_\mathrm{fusion} \cdot V_{\mathrm{plasma}}}{P_{\mathrm{heating}}} = \frac{5P_\alpha}{P_{\mathrm{heating}}} &}
\indent
where
\begin{enumerate}
  \item{$Q_\mathrm{phys}$=1 is break even}
  \item{$Q_\mathrm{phys}>$5 is a burning plasma}
  \item{$Q_\mathrm{phys}=\infty$ is an ignited plasma}
\end{enumerate}

\index{Engineering gain factor}
\noindent
The engineering gain factor~~\cite{authors}
\fla{ Q_\mathrm{eng} &= \frac{P_\mathrm{electricity}^\mathrm{out}}{P_\mathrm{electricity}^\mathrm{in}} &}


\index{Ignition condition}
\index{Lawson criterion|see{Ignition condition}}
\section{The Ignition Condition (or Lawson Criterion)}
The ignition condition describes the minimum values for density ($n$),
temperature ($T$), and energy confinement time ($\tau_E$) that are
required for a confined plasma to reach ignition.  Ignition is defined
as $P_\mathrm{alpha} > P_{\mathrm{loss}}$, where
$P_{\mathrm{auxilliary}} = 0$. \scite{wesson}{10-15} For a given
temperature, $T$, the following equations describe the minimum
$n\tau_E$ required to reach ignition under different assumptions:

\begin{enumerate}

   \item{$P_\mathrm{alpha} = P_\mathrm{conduction}$ \scite{wesson}{10-11}
     \fla{n\tau_E &= \frac{12kT}{\langle\sigma v\rangle E_\alpha} &}
    Using $\langle\sigma v\rangle_{DT} \approx 1.1\times10^{-24}T_\mathrm{keV}^2$
    and $E_\alpha = 3.5 \,\mathrm{MeV}$~~\cite{authors}:
    \fla{nT\tau_E &\gtrsim 3\times10^{21} \text{ m$^{-3}$ keV s} &}}

  \item{$P_\mathrm{alpha} = P_\mathrm{conduction} + P_\mathrm{bremsstrahlung}$~~\cite{authors}: 
     \fla{n\tau_E &= \frac{12kT}{\langle\sigma v\rangle E_\alpha - 2.14\times10^{-36}T_\mathrm{keV}^{1/2}} &}}

  \item{$P_\mathrm{alpha} = P_\mathrm{conduction} + P_\mathrm{bremsstrahlung}$ with alpha impurities $f_\alpha = n_\alpha/n_e$~~\cite{authors}: 
    \fla{n\tau_E &= \frac{12kT}{(1-2f_\alpha)^2\langle\sigma v\rangle E_\alpha - (1+2f_\alpha)2.14\times10^{-36}T_\mathrm{keV}^{1/2}} &}}

  \item{$P_\mathrm{alpha} = P_\mathrm{conduction} + P_\mathrm{bremsstrahlung}$ with impurity densities $f_j = n_j/n_e$~~\cite{authors}:
    \fla{n\tau_E &= \frac{12kT}{(1-\sum\limits_jf_jZ_j)^2\langle\sigma v\rangle E_\alpha - (2.14\times10^{-36})Z_\mathrm{eff}T_\mathrm{keV}^{1/2}} &}}

\end{enumerate}
