\chapter{Tokamak Edge Physics}
In this chapter, all units are SI with the exception of temperature,
which is defined in the historical units of eV (electron-volts).\\

\noindent
$e$ is the elementary electric charge\\
$q$ is the total particle charge\\
$Z$ is the particle atomic (proton) number\\
$T$ is the plasma temperature\\
$n$ is the plasma number density\\
$p$ is the plasma pressure\\
$e$ and $i$ refer to electrons and ions, respectively\\
$a$ and $R_0$ are the minor and major radii of a toroidal plasma\\

\index{Scrape off layer (SOL)}
\section{The Simple Scrape Off Layer (SOL)}
The simple SOL model describes 1D plasma flow from the core plasma to
material boundary surfaces for limited or diverted plasma along the
toroidal magnetic topology.  By assuming a high degree of
collisionality ($\nu_*$), fluid approximations for plasma flow are valid
and the neoclassical effects on particle orbits due to toroidal
magnetic topology can be safely ignored.\\

\index{Connection length}
\noindent
Parallel SOL connection length (rail/belt limiters, poloidal divertors) \scite{stangeby}{17}
\fla{ L_{||} &\approx \pi Rq&}

\noindent
Particle time in the SOL (simple 1D model) \scite{stangeby}{20}
\fla{ t_\mathrm{dwell} &\approx L_\parallel / c_\mathrm{s} &}

\noindent
SOL width (simple 1D model) \scite{stangeby}{23}
\fla{ \lambda_\mathrm{SOL} &\approx \left(D_\perp L_\parallel / c_\mathrm{s}\right)^{1/2} &}
\indent where $D_\perp$ is the anomalous diffusion coefficient.\\

\noindent
Force balance / pressure conservation in the SOL \scite{stangeby}{47}
\fla{ p_\mathrm{e} + p_\mathrm{i} + mnv^2 &= \mathrm{constant} &}

\noindent
Plasma density variation along the SOL \scite{stangeby}{47}
\fla{ n(x) &= \frac{n_0}{1+M(x)^2} &}
\indent where $n_0$ is the density at the 'top' of the SOL and \\
\indent $M=v/c_s$ is the plasma mach number.\\

\noindent
Electrons follow a Boltzmann distribution in the SOL \scite{stangeby}{28}
\fla{ n &= n_0\mathrm{exp}\left(eV/T_\mathrm{e}\right) &}

\noindent
SOL particle sources: ionization (i) and cross-field transport (t) \scite{stangeby}{35-40}
\fla{S_\mathrm{p} &= S_\mathrm{p,i}+S_\mathrm{p,t} = n_\mathrm{plasma}n_\mathrm{neutrals}\langle\sigma v\rangle_\mathrm{i} + D_\perp n / \lambda_\mathrm{SOL}^2 &}
\indent
where $\langle\sigma v\rangle_\mathrm{i} \equiv \langle\sigma v\rangle_\mathrm{i}(T_\mathrm{e},Z)$ is the ionization rate coefficient.\\

\noindent
Particle flux density in the SOL at the sheath edge (se) \scite{stangeby}{47}
\fla{ \Gamma_\mathrm{se} &= \frac{1}{2}n_0c_\mathrm{s} &}
\indent where $n_0$ is the density outside the pre-sheath.\\

\noindent
Electric field through SOL required to satisfy the Bohm Criterion \scite{stangeby}{48}
\fla{ V_\mathrm{se} &= -0.7\frac{T_\mathrm{e}}{e} &}

\noindent
Floating sheath voltage \scite{stangeby}{79}
\fla{ V_\mathrm{s} &= 0.5\frac{T_e}{e}\ln\left[2\pi \frac{m_e}{m_i}\left(Z+\frac{T_i}{T_e}\right)\right] &} %Changed \left(2+\frac{T_i}{T_e}\right) to \left(Z+\frac{T_i}{T_e}\right)

\noindent
Debye sheath width \scite{stangeby}{27}
\fla{ \lambda_\mathrm{Debye} &\approx \left(\frac{\epsilon_0 T_\mathrm{e}}{n_\mathrm{e}e^2}\right)^{1/2} &}


\index{Bohm criterion}
\section{Bohm Criterion}
The Bohm Criterion is derived from conservation of energy
($1/2m_iv^2=-eV$) and particle conservation ($n_iv=constant$).  In an
unmagnetized plasma it sets the SOL plasma exit velocity into the
sheath edge (se). In magnetized plasma, it sets the SOL plasma exit
velocity parallel to the magnetic field, after which the ions become
demagnetized and perpendicularly enter the sheath; electrons remain
magnetized. \scite{stangeby}{61-98}.\\

\noindent
Bohm Criterion (assuming $T_i = 0$) \scite{stangeby}{73}
\fla {v_\mathrm{se} &\geq \left(\frac{T_e}{m_i}\right)^{1/2} \!\! = c_s &}

\noindent
Bohm Criterion (general form) \scite{stangeby}{76}
\fla{ \int \limits_0^{\infty} \frac{f_\mathrm{se}^i(v)\,dv}{v^2} &\leq \frac{m_i}{T_e} &\\}

\index{Two point model}
\section{A Simple Two Point Model For Diverted SOLs}
Diverted plasmas can obtain significant $\Delta$T along the SOL,
resulting in divertor temperatures less than 10 eV.  The SOL can be
approximated using a two point model: point 1 is the outboard midplane
entrance to the SOL (``upstream'' or ``u'') and point 2 is the
divertor terminus of the SOL (``target'' or ``t'').  It is assumed
that upstream density, $n_u$, and the heat flux into the SOL,
$q_{||}$, are control parameters; upstream and target temperatures,
$T_u$ and $T_t$, as well as plasma density in front of the target,
$n_t$, are subsequently determined.

\subsection{Definitions}

\noindent
Dynamic and static pressure \scite{stangeby}{435}
\fla{ p &= nT\left(1+M^2\right) \hspace{0.5cm}\mathrm{where}\hspace{0.5cm} \left\{
  \begin{gathered}
    M_\mathrm{u}^2 \ll 1\\ %Changed \gg to \ll
    M_\mathrm{t}^2 \approx 1~\text{(Bohm Criterion)}\\
  \end{gathered} \right. 
  &
}

\index{Heat conduction}
\noindent
Heat conduction parallel to magnetic field \scite{stangeby}{187}
\fla{ q_{\parallel,~\mathrm{cond}} &= -k_0T^{5/2}\frac{dT}{dx} \hspace{0.5cm}\mathrm{where}\hspace{0.5cm} \left\{
  \begin{gathered}
    k_\mathrm{e,0} \approx 2000~~\mathrm{[W~m^{-1}~eV^{7/2}]}\\
    k_\mathrm{i,0} \approx 60~~\mathrm{[W~m^{-1}~eV^{7/2}]}\\
  \end{gathered} \right.
  &
}

\index{Heat flux tranmission coeficient}
\noindent
Sheath heat flux transmission coefficient at a biased surface \scite{stangeby}{652}
\fla{ \gamma &= 2.5\frac{T_i}{ZT_e}-\frac{eV}{T_e}+2\left[2\pi\frac{m_e}{m_i}\left(Z+\frac{T_i}{T_e}\right)\right]^{-1/2}\exp\left(\frac{eV}{T_e}\right) + \chi_i &}
\indent where $T_e\neq T_i$, $\chi_i$ is the electron-ion recombination energy,\\
\indent and no secondary electrons emitted.\\


 %Changed 2\left[\left(Z+\frac{T_i}{T_e}\right)2\pi\frac{m_e}{m_i}\right]^{1/2} to 2\left[2\pi\frac{m_e}{m_i}\left(Z+\frac{T_i}{T_e}\right)\right]^{-1/2}

\subsection{Fundamental Relations}
\noindent
SOL pressure conservation \scite{stangeby}{224}
\fla{ 2n_tT_t &= n_uT_u&}

\noindent
SOL power balance \scite{stangeby}{224}
\fla{ T_u^{7/2} &= T_t^{7/2} + \frac{7q_{||}L}{2 k_0 } &}

\noindent
SOL heat flux limited to sheath heat flux \scite{stangeby}{224} 
\fla{ q_{||} &= \gamma n_t T_t c_{st} &
             &\approx 7 \hspace{0.5cm} \text{(D-D plasma, floating surface)} &}

\subsection{Consequences}
\noindent
Upstream SOL temperature \scite{stangeby}{226}
\fla{ T_u &\approx \left(\frac{7q_{||}L}{2k_0}\right)^{2/7} \hspace{0.5cm} \text{assuming that } T_t^{7/2} \ll T_u^{7/2} &}
\indent
$\rightarrow T_u$ is independent of $n_u$\\
\indent
$\rightarrow T_u$ is insensitive to parameter changes due to the 2/7 power\\
\indent
$\rightarrow q_\parallel$ is extremely sensitive to $T_u$ due to the 7/2 power\\

\noindent
Target SOL temperature \scite{stangeby}{227}
\fla{ T_t &\approx \frac{2m_i}{\gamma^2e^2}\, \frac{q_\parallel^{10/7}}{(Lk_0)^{4/7}n_u^{2}} &}
\indent
$\rightarrow T_t$ is proportional to $\frac{1}{n_u^2}$\\

\noindent
Target SOL density \scite{stangeby}{227}
\fla{ n_T &= \frac{n_u^3}{q_{||}^2}\left(\frac{7q_{||}L}{2k_0}\right)^{6/7}\frac{\gamma^2 e^2}{4m_i}  &}
\indent
$\rightarrow n_T$ is proportional to $n_u^3$
